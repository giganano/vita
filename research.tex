
\documentclass[cv.tex]{subfiles}
\begin{document}

\category{Research}
%% METRICS %%%
\vspace{-5mm}
\noindent
{%
\newcommand{\verticallycentered}[1]{\begin{tabular}{c} #1 \end{tabular}}
\newcommand{\metric}[1]{
	\color{themecolor} \fontsize{25}{30} \selectfont #1 \vspace{3mm}
}
\newcommand{\PreserveBackslash}[1]{\let\temp=\\#1\let\\=\temp}
\newcolumntype{C}[1]{>{\PreserveBackslash\centering}p{0.17\textwidth}{#1}}
\newcolumntype{R}[1]{>{\PreserveBackslash\raggedleft}p{0.17\textwidth}{#1}}
\newcolumntype{L}[1]{>{\PreserveBackslash\raggedright}p{0.17\textwidth}{#1}}
\begin{center}
\begin{tabular}{|C|C|C|C|C|}
\metric{~32} &
\metric{~12} &
\metric{~20} &
\metric{~750+} &
\metric{~18}
\\
\verticallycentered{\textbf{Journal} \\ \textbf{Publications}} &
\verticallycentered{\textbf{1st \& 2nd} \\ \textbf{Author}} &
\verticallycentered{\textbf{Contributing} \\ \textbf{Author}} &
\verticallycentered{\textbf{Citations}} &
\verticallycentered{\textbf{H-Index}}
\end{tabular}
\end{center}
}

%%% INTERESTS %%%
\vspace{2mm}
\noindent
{\color{themecolor} \large Interests}
\par\noindent
Galactic chemical evolution -- The Milky Way -- Dwarf galaxies -- The
astrophysical origin of the elements -- Big bang nucleosynthesis -- Near field
cosmology -- Astronomical software

%%% ADS LIBRARIES %%%
\vspace{5mm}
\noindent
{\color{themecolor} \large NASA ADS Libraries}
\ifthenelse{\boolean{includepubs}}{%
	\small
	(A full list of my journal publications is included.)
}{%
	\small % preserves spacing between the two versions of the document
}
\par\noindent
\parbox{0.18\textwidth}{%
	\small\raggedleft
	All My Papers \par
	1st \& 2nd Author \par
	Co-Author \par
}
\hspace{1mm}
\parbox{0.8\textwidth}{%
	\small\vspace{1mm}
	\url{https://ui.adsabs.harvard.edu/public-libraries/rIqfpNKmSdaOMIAhkk2VzQ}
	\par
	\url{https://ui.adsabs.harvard.edu/public-libraries/go1WSseGTMeft2SxdESAgw}
	\par
	\url{https://ui.adsabs.harvard.edu/public-libraries/sZkjSf_XRSKSRykqBe6B_w}
}

%%% TALKS %%%
\vspace{5mm}
\noindent
{\color{themecolor} \large Seminars \& Conference Presentations}
\vspace{1mm}
\par\noindent
\parbox{0.18\textwidth}{%
	\raggedleft
	Contributed Talk \par
	Invited Seminar \par
	Invited Seminar \par
	Invited Seminar \par
	Poster \par
	\null \par
}
\hspace{1mm}
\parbox{0.8\textwidth}{%
	\vspace{1mm}
	\textbf{Sloan Digital Sky Survey Collaboration Meeting} \hfill 2025 \par
	\textbf{University of California, Davis} (Davis, CA) \hfill 2025 \par
	\textbf{Stockholms Universitet}, Dept. of Astronomy (Stockholm, Sweden)
	\hfill 2025 \par
	\textbf{Uppsala Universitet}, Dept. of Physics \& Astronomy
	(Uppsala, Sweden) \hfill 2025 \par
	\textbf{Small Galaxies, Cosmic Questions - II} \hfill 2024 \par
	University of Durham (Durham, United Kingdom) \par
}

\newpage
\noindent
\parbox{0.18\textwidth}{%
	\raggedleft
	Contributed Talk \par
	\null \par
	Contributed Talk \par
	Invited Seminar \par
	Contributed Talk \par
	\null \par
	Contributed Talk \par
	\null \par
	Dissertation Talk \par
	Contributed Talk \par
	Contributed Talk \par
	Contributed Talk \par
	Poster \par
	Invited Seminar \par
}
\hspace{1mm}
\parbox{0.8\textwidth}{%
	\vspace{1mm}
	\textbf{DHWFEST: Dark, Hot, Warm, and Fuzzy mattEr in Space and Time}
	\hfill 2024 \par
	University of Utah (Salt Lake City, UT) \par
	\textbf{Sloan Digital Sky Survey Collaboration Meeting} \hfill 2024 \par
	\textbf{Lund University}, Dept. of Physics (Lund, Sweden) \hfill 2024 \par
	\textbf{ADONIS: Abundance Gradients in the Local Universe}
	\hfill 2024 \par
	Munich Institute for Astro-, Particle, and BioPhysics (MIAPbP)
	(Munich, Germany) \par
	\textbf{Surveying the Milky Way: The Universe in Our Own Backyard}
	\hfill 2023 \par
	California Institute of Technology (Pasadena, CA) \par
	\textbf{241$^\text{st}$ American Astronomical Society Meeting}
	\hfill 2023 \par
	\textbf{Sloan Digital Sky Survey Collaboration Meeting}
	\hfill 2021 \par
	\textbf{Galactic Archaeology with Hermes Science Meeting}
	\hfill 2021 \par
	\textbf{Sloan Digital Sky Survey Collaboration Meeting}
	\hfill 2020 \par
	\textbf{236$^\text{th}$ American Astronomical Society Meeting}
	\hfill 2020 \par
	\textbf{University of California, Santa Cruz} (Santa Cruz, CA) \hfill 2019
	\par
}



%%% SOFTWARE DEVELOPMENT %%%
\vspace{4mm}
\noindent
{\color{themecolor} \large Astrophysical Software Development}
\par\noindent
\parbox{0.35\textwidth}{%
	\centering
	\includegraphics[scale = 0.33]{vice-logo.png}
}
\parbox{0.63\textwidth}{%
	\textbf{Versatile Integrator for Chemical Evolution (VICE)} \par
	Lead developer and license owner (Spring 2018 -- Present) \par
	Documentation: \url{https://vice-astro.readthedocs.io} \par
	Source Code: \url{https://github.com/giganano/VICE.git} \par
	Install: \url{https://pypi.org/project/vice}
}

%%% Observing Programs %%%
\vspace{4mm}
\noindent
{\color{themecolor} \large Observing Programs}
\par\noindent
\textbf{PI}: \textit{The First Extragalactic Measure of the Helium Isotopic
Ratio -- A New Test of Fundamental Physics}
\par \noindent
\parbox{0.18\textwidth}{%
	% \vspace{2mm}
	\raggedleft
	2024B \par
	2025A
}
\hspace{1mm}
\parbox{0.8\textwidth}{%
	WINERED spectrograph, 18 hours (Clay 6.5-m Telescope, Las Campanas
	Observatory) \par
	MIKE spectrograph, 6 hours (Clay 6.5-m Telescope, Las Campanas Observatory)
}

\end{document}
